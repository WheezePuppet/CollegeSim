
\section{RESULTS}
\label{sec:results}

Of course, this simulation has an extensive list of parameters and moving parts, which means that, by necessity, 
there are nearly endless possibilities for analysis. More than twenty variables were left to be varied at the command line, 
each of them able to take on an infinite number of values. To direct our study, we decided to focus on the effects of the 
importance of race (the race weight) on the simulation as well as the potential effects of implementing various policies, 
which included mandated interracial friendships and the construction of forcibly integrated groups.

First, it is important to inspect the graphs of the average number of friends held by minorities and whites across the years 
in the simulation. Note that for any nonzero race weight, minorities had fewer friends on average than whites, 
regardless of other parameter settings. That is, if race matters even slightly to the students in the simulation, then 
minorities will form fewer relationships. When race is not a factor in forming relationships, the races have an approximately 
equal average number of friends, as expected.

By creating stacked histograms of the racial composition of each group, we observed that the composition of groups at the null 
race weight matched the expected composition, which was based on the population composition, more closely than any of the 
nonzero race weights. As race weight increased, the observed group composition grew farther from that which was expected. In 
fact, at the most extremely high values of race weight, an increasing proportion of the groups consisted only of white 
students, as minorities lose affinity towards many groups, and thus can neither join nor remain in them.

We used boxplots to track and display the frequency of perceived similarity levels whenever two students encountered each 
other. For the null race weight, the interracial encounters experienced the same perceived similarity levels as the 
encounters between two students of the same race. These similarity ratings spread between 0.8 and 1.0. For higher values of 
race weight, the homogeneous race interactions remained at approximately this level. Meanwhile, the perceived similarity of 
heterogeneous race interactions creeped lower. Because of this, homogeneous encounters resulted in friendship slightly more 
often than heterogeneous encounters as race weight was given positive value. As the importance of race increased, the number 
of heterogeneous encounters which resulted in friendship expectedly decreased. However, the number of homogeneous encounters 
which resulted in friendship stayed approximately the same.

After preliminary analyses utilizing various graphs, we proceeded to analyze the data in a more concrete manner. For each 
parameter set, the simulation was run for ten trials, and information about each student was recorded for every year of the 
simulation. We transformed this output into two vectors, one documenting the white students and another documenting the 
minority students. Each element of the vector corresponded to the proportion of that individual's friends which were 
minorities. To prevent duplicates, since a student who attended the college for four years would be included in the file four 
times, we considered only the data that was obtained during a student's senior year. Additionally, it is important to note 
that having no minority friends when you have some number of friends is potentially much different from having no 
minority friends when you have no friends in general. So, we also omitted the data where a student had zero total friends. 
After creating a vector for each trial run, we concatenated the ten vectors. The resultant vectors were around 160,000 in 
length for the white data and around 40,000 in length for the minority data.

For a population which is integrated, these vectors would have true mean equal to the population proportion of minorities.
For each trial, this proportion was 0.2. This information was obtained by performing a t-test on the vectors of data. As 
expected, under the null race weight, the t-test returned insignificant for both the white and minority vectors, meaning 
there was not evidence to reject the null hypothesis that the true mean was indeed 0.2. Of course, eliminating race as a 
consideration should in fact contribute to a perfectly integrated community. For any positive race weight, we fail to have 
an integrated campus. That is, the t-tests performed on data collected at all other race weights rejected the null hypothesis 
that the true mean is 0.2. Even when race weight was as low as 1, the campus experienced significant segregation.

A second set of vectors was generated using the same parameter sets as the first, but with the exception that we have ``turned 
on" a policy. An inactive policy, one whose parameter was previously zero, was set to have some nonzero parameter. Primarily, 
we investigated the potential effects of creating five intentionally integrated groups at the beginning of each year. We then 
determine if there is integration after this policy is implemented by the same method as before. Additionally, we perform 
t-tests on the sets of vectors - whites when policy is off with whites when policy is on, and minorities when policy is off 
with minorities when policy is on - to determine whether or not the policy had a measurable effect on the population. 

For the null race weight, we of course still have integration after the implementation of the policy, and the policy does not 
demonstrate any significant effect on the population. The policy also did not have an effect when race weight was 1, leaving 
the campus still as segregated as it was before the policy was implemented. These results were repeated when race weight was 
set to 10. When the weight of race was increased to 20, both populations were segregated before as well as after the policy 
was implemented. However, the white population did have a significant chance after the policy came ``on," and that was that 
the mean proportion of friends went down, from 0.1646 to 0.1634. So, in this case, it appears that the policy may have even 
increased segregation. Finally, for the extreme setting of race weight at 100, the populations were segregated at both 
collection points. The policy had a significant effect on the white data, decreasing the mean from 0.0935 to 0.0926.
%I left out that there was a non-significant (but almost, p=0.07748) effect on the minority data where the mean
%actually went up, so it's almost like the policy made things worse

We also briefly investigated the effects of using ten integrated groups instead of five. When race weight equalled 1, the 
t-test on the mean of the minority data returned insignificant, which means that the true mean is likely to be 0.2. In other 
words, after this more extreme policy implementation, the white population remained segregated, but the minority population 
had induced integration. Despite the white population remaining segregated, the t-tests between the sets of data revealed 
that in both the white and minority cases, the policy implementation made a difference. The actual difference may 
be unexpected: The policy caused the mean proportion of minority friends held by whites to {\emph decrease}, from 0.1983 to 
0.1974. This is evidence that the policy actually caused higher segregation for whites. However, the positive result is that 
the mean proportion of minority friends held by minorities also decreased, from 0.2025 to 0.2009. This means that, on 
average, minorities were making more friends with whites, improving the integration from their direction.
%If a minority has fewer minority friends, they have more white friends, which means they experience more
%integration. Does that really help them at all? If I'm a minority, am I better off under a policy that causes me
%to have more white friends than I would have been otherwise? Or would it not be as much of an improvement as if I just had
%more friends in general? Or, would I feel more accepted if I could make lots of friends with people of my own race? Is the
%end game to make whites have more minority friends, or to make minorities have more white friends? Which is truly the most
%beneficial in terms of making minorities feel more accepted? Stephen?

%I could see the argument for "minorities having more friends that are white" being beneficial from the viewpoint that
%then they may not feel as threatened by other whites they encounter (may feel more at home/relaxed) as well as that the 
%spread of ideas due to interaction with people who are different is increased.

Finally, we momentarily examined the effects of using a different policy - forced interracial friendships - on the 
populations. We set the number of forced interracial friendships to be 10, a value which is relatively high, since the average 
total number of friends is only between about 10 to 20. Again, at race weight set equal to 1, the white population 
experienced segregation, while the minority population experienced integration. In both cases, the policy had an effect on the 
population, again decreasing the mean proportion of minority friends held by both groups. When this was repeated for race 
weight set to 20, the populations were segregated both before and after the policy was implemented, and the policy appears 
to have had no effect.

Clearly, there is a wealth of analysis to be done on this simulation, and we have barely even brushed the surface of the 
possibilities. In future analyses, we hope to collect data on more combinations of the parameters. We seek to determine how 
extreme a policy must be for it to effectively integrate the population for a variety of race weights. Obviously, the null 
weight for race displayed integration regardless of policy status. Other than that, we failed to achieve pure integration 
with any one other policy. It may be that the policies need be more severe, or that we must use the policies in 
conjunction. It also may be that the policies struggle at this proportion of minorities in the population, but would 
excel if the number of minorities was marginally increased. There are an innumerable number of questions to be posed 
surrounding this simulation, and an even wider expanse of data to be collected.
