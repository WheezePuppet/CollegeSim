\section{MOTIVATION}
\label{sec:intro}

Diversity, and in particular racial diversity, is one of the most oft-cited
goals of institutions in higher education. Not only is a racially balanced
student body evidence of fairness in admission practices, but the experiences
of all students -- white and minority alike -- are said to benefit from
exposure to and relationships with peers of different ethnicities and
cultures. These benefits range from the simple appreciation and respect for
people of different races and backgrounds to deeper concerns such as civic
responsibility, identity construction, and strong cognitive
growth \cite{gurin_diversity_2002}.

The goal is not easy to attain. College enrollment rates for racial minorities
(Asian-Americans excepted) are disproportionately low across the country, and
have been for many years \cite{ryu_minorities_2010}. One particularly difficult
aspect of the problem is attrition (dropout rate), which is consistently
higher for minorities than for white students \cite{zea_predicting_1997}. If a
university cannot retain its minority students once they are admitted, any
improvements in recruitment will be futile. An important question thus
arises: why \textit{do} minorities drop out with greater frequency? Contrary
to what some might think, the problem is not mainly academic. Studies differ,
but the proportion of withdrawals due to academic problems seems to range from
around 15\% \cite{kalsner_issues_1991} to somewhat less than
half \cite{suen_alienation_1983}. There are clearly major contributors to
attrition other than academic performance.

One oft-cited contributor is the phenomenon of
``alienation" \cite{burbach_development_1972,dean_alienation:_1961}, loosely
defined as a feeling of exclusion or non-belonging. Minorities at
predominantly white institutions have significant psychological barriers to
overcome. This has been well-documented at both the high
school \cite{calabrese_structure_1988} and college \cite{nora_role_1996} levels.
Kalsner notes that integration into a ``largely white environment" can be
especially difficult for black college students, and Loo states that ``even if
minority students show high levels of academic satisfaction, they may feel
socially and culturally alienated." \cite{loo_alienation_1986} Zea's study
concludes, not surprisingly, that ``when students perceived the environment as
unwelcoming because of race, ethnicity, or religion, their desire to continue
attending college diminished. Ethnic minority students were more likely to
report experiencing disrespect." \cite{zea_predicting_1997}

Looking more carefully at the phenomenon of alienation, Dwight Dean identified
at least three dimensions: feelings of ``meaninglessness" (lack of direction
and purpose in life), ``powerlessness" (lack of control over one's life), and
``social estrangement" (or feelings of loneliness)
\cite{dean_alienation_1961}. One particularly compelling study of black
students at a predominantly white university \cite{suen_alienation_1983} found
that in order to mitigate feelings of alienation among minority college
students, it is this third dimension that is most important. Apparently, the
perception of social estrangement -- feeling isolated or disconnected from the
larger population -- is a key contributing factor to the attrition rate for
minority college students. Institutions that value racial diversity must find
ways to address this widespread problem.

