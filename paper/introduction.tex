\section{MOTIVATION}
\label{sec:intro}

\subsection{Factors in Racial Diversity}

Achieving diversity, and in particular racial diversity, is one of the most
oft-cited goals of institutions in higher education. Not only is a racially
balanced student body evidence of fairness in admission practices, but the
experiences of all students -- white and minority alike -- are said to benefit
from exposure to and relationships with peers of different ethnicities and
cultures. These benefits range from the simple appreciation and respect for
people of different races and backgrounds to deeper concerns such as civic
responsibility, identity construction, and strong cognitive growth
\cite{gurin_diversity_2002}.

The goal is not easy to attain. College enrollment rates for racial minorities
(Asian-Americans excepted) are disproportionately low across the United
States, and have been for many years \cite{ryu_minorities_2010}. One
particularly difficult aspect of the problem is attrition (dropout rate),
which is consistently higher for minorities than for white students
\cite{zea_predicting_1997}. If a university cannot retain its minority
students once they are admitted, any improvements in recruitment will be
futile. An important question thus arises: why \textit{do} minorities drop out
with greater frequency? Contrary to what some might think, the problem is not
mainly academic. Studies differ, but the proportion of withdrawals due to
academic problems seems to range from around 15\% \cite{kalsner_issues_1991}
to somewhat less than half \cite{suen_alienation_1983}. There are clearly
major contributors to attrition other than academic performance.

One oft-cited contributor is the phenomenon of ``alienation"
\cite{burbach_development_1972,dean_alienation:_1961}, loosely defined as a
feeling of exclusion or non-belonging. Minorities at predominantly white
institutions have significant psychological barriers to overcome. This has
been well-documented at both the high school \cite{calabrese_structure_1988}
and college \cite{nora_role_1996} levels. Kalsner notes that integration into
a ``largely white environment" can be especially difficult for black college
students, and Loo states that ``even if minority students show high levels of
academic satisfaction, they may feel socially and culturally alienated"
\cite{loo_alienation_1986}. Zea's study concludes, not surprisingly, that
``when students perceived the environment as unwelcoming because of race,
ethnicity, or religion, their desire to continue attending college diminished.
Ethnic minority students were more likely to report experiencing disrespect."
\cite{zea_predicting_1997}

Looking more carefully at the phenomenon of alienation, Dwight Dean identified
at least three dimensions: feelings of ``meaninglessness" (lack of direction
and purpose in life), ``powerlessness" (lack of control over one's life), and
``social estrangement" (or feelings of loneliness)
\cite{dean_alienation:_1961}. One particularly compelling study of black
students at a predominantly white university \cite{suen_alienation_1983} found
that in order to mitigate feelings of alienation among minority college
students, it is this third dimension that is most important. Apparently, the
perception of social estrangement -- feeling isolated or disconnected from the
larger population -- is a key contributing factor to the attrition rate for
minority college students. Institutions that value racial diversity must find
ways to address this widespread problem.

\subsection{Social Phenomenon: Propinquity}

Playing out against this backdrop of social estrangement are the dynamics of
friendship formation and dissolution, which in turn are based largely on two
immemorial principles. The first is that of \textbf{propinquity}: the physical
proximity between two people, or other factors which affect how often they
meet each other by chance\cite{festinger_social_1950}. Simply put, the more
often two people encounter one another, the more likely it is they will form a
friendship. And for those who have already become friends, maintaining a
steady frequency of contact is important for the relationship to be
maintained.

As far as mixed-race relations are concerned, it was demonstrated decades ago
\cite{nahemow_similarity_1975} that people of different races tend not to form
friendships unless they live in very close proximity to each other. Thus it is
crucial that if racial segregation is to be overcome in a campus environment,
students of different races simply encounter one another often. In college
residence halls in particular, closeness of physical location has been shown
to be a good predictor of attraction \cite{priest_proximity_1967}, lending
weight to this argument.

\subsection{Social Phenomenon: Homophily}

Even more basically, human beings generally prefer to form friendships with
``other people like me," at least on some dimensions. This is the
\textbf{homophily} principle: ``likes attract". Sociologists have long noted
that people with similar traits (whether physical, cultural, or attitudinal)
interact with one another more often than with dissimilar. 
\cite{centola_homophily_2007}.

This is the racial diversity problem in its most basic form: students want to
be friends with other students like them, and when a white encounters a
minority, race is obviously a very visible \textit{non}-likeness immediately
evident to both parties. Indeed, although homophily is a general feature that
applies to many attributes, it is race and ethnicity that more strongly divide
our social environments than any other \cite{mcpherson_birds_2001}. Such an
effect must be overcome by other factors if friendship is to occur. To make
matters worse, since students of different races tend to segregate themselves
by race, they have relatively fewer chances to meet and have that possibility.

As with propinquity, this concept affects not only the likelihood of people
becoming friends, but the chances of their maintaining that friendship;
McPherson \textit{et al.} note that ``ties between nonsimilar
individuals...dissolve at a higher rate." \cite[p.415]{mcpherson_birds_2001}
Also, homophily can ``flow" in the opposite direction: in addition to ``choice
homophily," in which people choose to interact with those who are perceived to
be already similar to them, ``induced homophily" emerges from influence
dynamics that make friends more similar over time
\cite{mcpherson_homophily_1987}. Clearly, then, homophily can become
self-perpetuating, as people choose friends based on perceived similarity and
then are further influenced by them. In the case of racial segregation, this
makes it all the more important for mixed-race encounters to happen early in a
student's college career.

% "homogeneity within groups is the overwhelming determinant of homophily"
% (McPherson & Smith-Lovin 1987 - do we need to cite this?)

\subsection{Simulating Campus Racial Diversity}

The network of friendships that exists on a college campus is the result of,
among other things, the innumerable chance encounters, introductions,
impressions, and choices to form relationships that have occurred over many
years. It is the product of many generations of different students, the social
groups and cliques they have formed and disbanded, the ways they perceived
each other when they met, and the ways they have jointly influenced each other
afterwards. The formation of a single mixed-race friendship early in a
student's career may influence him or her in ways that make it likely to form
others, and thus propagate to affect multiple peers and groups. Clearly, no
matter how we might choose to measure ``segregation," we cannot predict it
analytically.  

In this paper, we describe an agent-based simulation to model these
interrelationships and influences. Software objects, each representing an
individual student, possess a set of malleable attributes and also a ``race"
trait. They encounter each other randomly, decide whether or not to form
friendships (using homophily as a key but not decisive factor), influence the
attributes of their friends, and dissolve the friendships which are not
``refreshed" often enough by additional future encounters. Our goals are to
create a useful abstraction for the domain of peer influence on a college
campus, to experiment and establish the quantitative effects and boundaries of
key parameters, to validate the model against real-world empirical data, and
to ascertain the possible effects of certain policies that institutions might
choose to enact to encourage racial integration. Our work is still at a
preliminary stage, and these goals are only partially realized in this paper.
