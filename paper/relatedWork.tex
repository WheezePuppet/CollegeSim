
\section{RELATED WORK}
\label{sec:related}

The first and most famous agent-based model of racial segregation is of course
Thomas Schelling's pioneering work \cite{schelling_models_1969}, implemented
even before computing power was commonly available. Agents living in a grid of
housing plots (similar to a cellular automaton, and simulated with coins on
paper by Schelling) periodically examine the racial makeup of their neighbors,
and decide to move (randomly to a new location in the grid) if the fraction of
same-race neighbors falls below a certain threshold. His landmark finding was
that even when agents have only very mild preferences to live near ``like"
neighbors, total segregation of a residential community is likely to result.
Schelling's original work has inspired extensions of this general theme that
continue to the present day \cite[to name just a
few]{laurie_role_2003,chen_emergence_2005,collard_emergence_2013,pate_segregation_2010,yin_dynamics_2009,fossett_overlooked_2005,abbas_agent-based_2013,bischi_adaptive_2011}.

Junfu Zhang \citeyear{zhang_evolutionary_2002} strengthened this claim and
discovered that even in a model where all agents \textit{desire} to have
mixed-race neighbors (rather than simply being willing to tolerate them),
total segregation will still occur. This underscores just how inherent the
segregation of agents is to complex systems, and suggests that any measures
taken to correct it must be drastic. Indeed, as Clark and Fossett state
\citeyear{clark_understanding_2008}, ``mere tolerance and the absence of
virulent housing discrimination will not produce integration under the
prevailing patterns of ethnic preference, at least not in the short run."

Gretchen Koehler \citeyear{koehler_racial_2001} critiqued the Schelling
model's applicability to the university housing problem, observing that
empirically, dissatisfied college students do not usually request housing
reassignment as the Schelling model would require
\cite{koehler_residential_2010}. Our model accordingly does not involve
students transferring from their initially assigned (mixed-race) residential
assignment, as described in the next section. Interestingly, Koehler also
found through surveys that both white and minority students express a
preference for integrated residence halls, although this preference is much
stronger for the minorities.

Kathleen Carley's ``Construct" system
\cite{carley_theory_1991,schreiber_construct-multi-agent_2004} was an early,
powerful agent-based simulation framework for modeling complex social and
behavioral dynamics. It has been used, among other things, for investigating
the effects of homophily on relationship formation and whether or not
homophily alone is sufficient to produce the kind of tiered social network
seen in practice \cite{hirshman_leaving_2011}. (It isn't.)

Regarding the social influence of friends upon one another, one important work
is Robert Axelrod's agent-based model of cultural diffusion
\cite{axelrod_dissemination_1997}, which incorporates interaction effects
between agents' attributes. ``\dots the likelihood that a given cultural
feature will spread from one individual (or group) to another depends on how
many other features they may already have in common." (p.205) Axelrod's work
does not address segregation with respect to a fixed attribute (race), but
instead examines the number and nature of stable cultural regions, with some
surprising results (for instance, that moderately-sized territories have more
distinct stable regions than either larger or smaller ones do). One key
difference between this model and ours is that in Axelrod's model, agents only
interact with their immediate geographic neighbors, which limits the number of
paths through which influence can propagate. In our model, students are not
restricted to interacting only on a fixed grid, but instead can form any
number of overlapping groups, and indeed meet each other completely randomly
with a certain probability. Our aim is to explore how the ``messy" structure
of social networks plays out in a setting of multiple races and mutual
influence.

A more recent model of mutual social influence \cite{schuhmacher_using_2014}
focused on risky vs. conventional behaviors in adolescents, and the ways in
which behavioral choices propagated in a social network with both dyadic and
group interactions. The authors had great success reproducing stylized facts
from the adolescent development literature, confirming that an agent-based
approach can indeed capture the key dynamics here. They did not, however,
study racial factors or effects.

Ghasem-Aghaee and Oren \citeyear{ghasem-aghaee_cognitive_2007} give a detailed
and principled model of personality attributes and the ways in which they
develop over time. Unlike our scheme and those of Axelrod and others
\cite{epstein_growing_1996}, which assign to agents generic ``attributes" as
placeholders for unspecified real-world qualities, Ghasem-Aghaee's model is
based on contemporary psychology research and simulates specific personality
traits such as openness, conscientiousness, and extraversion. Adopting such an
approach in future work may help us refine our model to more closely imitate
specific personality effects. Ghasem-Aghaee's simulation, however, does not
explicitly model the relationships in an evolving social network, as ours
does.

We are unaware of attempts to simulate institutional policies regarding race
in higher education. Feitosa \textit{et al.}
\citeyear{feitosa_multi-agent_2011} pursued a similar aim in the domain of
urban segregation, and express cautious optimism about their simulator's
utility in determining the range of impact of anti-segregation policies.

Other simulations have focused on inter-school segregation, rather than
intra-school. Researchers \cite[for instance]{stoica_schelling_2014,millington_aspiration_2014} have adapted
Schelling's model to scenarios in which parents choose the school they will
send their children to, in part based on its proximity to their residence, its
reputation, and its racial composition. This is very different from our work,
which analyzes the social relationships between students of different races
within a single institution.

