
\section{DISCUSSION AND CONCLUSION}
\label{sec:discussion}

We have presented a principled model of a college social environment, based on
findings from the social psychology literature on friendships, groups, race
relations, and campus culture. We have verified the model and demonstrated
that the basic effects we hoped to see realized do materialize as expected. We
are now in a position to use the simulation to explore the impact of various
policies and parameters.

In future analyses, we hope to collect data on more combinations of the
parameters. We seek to determine how extreme a policy must be for it to
effectively integrate the population for a variety of race weights. Our
cursory examination thus far revealed that although both of our policies had a
noticeable effect, neither is sufficient to completely overcome segregation.
It may be that the parameter settings need to be more extreme, or that we must
use the policies in conjunction. It also may be that with a small proportion
of minorities in the population the policies are unsatisfactory, but would
excel if the number of minorities was marginally increased. 

CollegeSim is open source, and is available for browsing or download at
\url{https://github.com/WheezePuppet/CollegeSim}. It was written in Java
using the MASON agent-based toolkit \cite{luke_mason:_2005}. The API can be
browsed at \url{http://rosemary.umw.edu/~stephen/CollegeSimJavaDoc}.

An online demo of CollegeSim is available at
\url{http://caladan.umw.edu:22223}, and we encourage readers to experiment
with parameter settings and run the simulation themselves.

%Reminder: Cite Currarini, see "our d.v." email.
%Reminder: Cite Massey and James, see Issue #17
