
\section{CONCLUSION}
\label{sec:discussion}

We have presented a principled model of a college social environment, based on
findings from the social psychology literature on friendships, groups, race
relations, and campus culture. We have verified the model and demonstrated
that the basic effects we hoped to see realized do materialize as expected. We
are now in a position to use the simulation to explore the impact of various
policies and parameters.

In future analyses, we hope to collect data on more combinations of the
parameters. We seek to determine how extreme a policy must be for it to
effectively integrate the population for a variety of race weights. Our
cursory examination thus far revealed that although both of our policies had a
noticeable effect, neither is sufficient to completely overcome segregation.
It may be that the parameter settings need to be more extreme, or that we must
use the policies in conjunction. It also may be that with a small proportion
of minorities in the population the policies are unsatisfactory, but would
excel if the number of minorities was marginally increased. 

True validation of the model will be challenging, as not all relevant
parameters can be ascertained from existing empirical data. For instance, the
number of friendships students have with students of the same or different
race is largely inaccessible except through surveys. In addition, it will be
difficult to find a real-world institution that has implemented policies like
\textbf{InitialMixedRaceDyads} and \textbf{OrientationGroups} and has data for
both before and after the implementation. We plan to approach these challenges
in creative ways in our future work, to attempt to establish confidence that
the model's behavior resembles that of real-world institutions.

\href{https://github.com/WheezePuppet/CollegeSim}{CollegeSim} \cite{CollegeSim} is open source, and is available for browsing or download. It was written in Java
using the MASON agent-based toolkit \cite{luke_mason:_2005}.

A \href{http://caladan.umw.edu:22223}{web demo} \cite{WebDemo} of CollegeSim is available, and we encourage readers to experiment
with parameter settings and run the simulation themselves.

%Reminder: Cite Currarini, see "our d.v." email.
%Reminder: Cite Massey and James, see Issue #17
