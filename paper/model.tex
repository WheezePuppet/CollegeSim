
\section{THE MODEL}
\label{sec:model}

% Hand-tweaking the cite so as to avoid spitting out 14 author names. -SD
We present the model for the CollegeSim simulation using an abbreviated
version of the ODD protocol \cite[Grimm \textit{et al.} 2008]{polhill_using_2008}.

\subsection{Purpose}

CollegeSim is an agent-based model designed to simulate an evolving social
network among college students. In addition to their other attributes, Student
agents are assigned one of two ``races" (white or minority) which factors into
their perceived similarity to other agents they encounter, and accordingly to
their likelihood of forming friendships with them. The ultimate goal of the
model is to investigate how racial segregation -- defined as the average
proportion of friendships students have with others of the same race -- is
affected by the strength (importance) of the race trait, the speed with which
students adapt their preferences, and the presence or absence of certain
institutional policies designed to influence segregation.

\subsection{Entities, State Variables and Scales}

The model has two kinds of entities: Students and Groups.

%%%%%%%%%%%%%%%%%%%%%%%%%%%%%%%% Students %%%%%%%%%%%%%%%%%%%%%%%%%%%%%%%%

\vspace{.1in}
Each \textbf{Student} agent has the following attributes:

\begin{description}
\itemsep.1em
\item[ID] A unique ID number.

\item[Race] Either ``white" or ``minority."

\item[Year] An integer from 1 to 4 indicating the Student's current year
in school (freshman, sophomore, junior, and senior). For simplicity the model
does not allow larger numbers (``fifth-year seniors"); any Student reaching
the end of their 4th year % without having dropped out 
automatically graduates.

\item[Preferences] An array (of configurable size) of real numbers (between 0
and 1) representing the student's abstract preferences. None have fixed
interpretations, but as a conceptual example one might consider preference \#9
to represent ``political persuasion, on a scale from extremely liberal (0) to
extremely conservative (1)." Preferences are strictly independent of each
other in the sense that a high value for one does not imply a high or low
value for any other.

A Student's preferences (and hobbies; see below) will change throughout the
simulation in response to their friends. This is in line with Axelrod's
definition of an agent's ``culture" as ``a set of individual attributes that
are subject to social influence" \citeyear{axelrod_dissemination_1997}.

\item[Hobbies] An array (of configurable size) of real numbers (between 0 and
1) representing how much of the student's time is dedicated to a particular
activity. Again, none have fixed interpretations (hobby \#17 might be thought
of as ``the proportion of time the Student spends playing Pok\'emon"), but
they are \textit{not} independent of each other since a greater value for one
implies a lesser value for the others. Every time a Student's hobby value
changes, all of that Student's hobby values are recalibrated proportionately
so that they sum to 1.

\item[Friends] An array of references to other Student objects with whom
it has active relationships. This array will grow and shrink over time as new
friendships are formed and ``non-refreshed" friendships expire.

\end{description}

Note that Students do not have spatial coordinates; their ``position" in the
simulation is defined solely by their friendships with each other and their
membership in Groups.

\vspace{.2in}
%%%%%%%%%%%%%%%%%%%%%%%%%%%%%%%%%%%% Groups %%%%%%%%%%%%%%%%%%%%%%%%%%%%%%%%

\textbf{Group} entities represent the abstract social groups (corresponding to
anything from common residential hallways to student clubs to simple informal
cliques) that Students form with each other. Groups impact the simulation in
two ways: (1) a Group's members will jointly influence each other's preference
and hobby attributes, and (2) Students interact with those in their groups
more often than with the population at large.

\vspace{.1in}
Each Group agent has the following attributes:

\begin{description}
\itemsep.1em

\item[ID] A unique ID number.

\item[IsFixed] A logical (boolean) value indicating whether or not the
group's membership is permanently fixed, or whether it can
fluctuate over time. This is used to implement certain policies (see below).

\item[RecruitmentFactor] A real number (from 0 to 1) indicating how
aggressive the group is in attracting new members. (See below for
interpretation.)

\item[Members] An array of references to the Students who comprise its
membership.

\end{description}

There are also two Policies in the model, \textbf{OrientationGroups} and
\textbf{InitialMixedRaceDyads}, which can be separately enabled (see below).



%%%%%%%%%%%%%%%%%%%%%%%%%%%%%%%% Schedule %%%%%%%%%%%%%%%%%%%%%%%%%%%%%%%%%%

\subsection{Process Overview and Scheduling}

The simulation progresses according to an idealized academic calendar. Each
academic year is comprised of nine academic months (August through April) and
three summer months. Student agents and Group agents carry out their actions
in each academic month: all students, followed by all groups. Additionally, a
high-level controller called ``Sim" carries out activities at the start of
every academic year (immediately before all Students run in August) and at its
end (immediately after all Groups run in April). See the ``Submodels" section,
below, for details about each step.

\begin{enumerate}
\itemsep.1em
\item At the start of each academic year (pre-August):

    \begin{enumerate}
    \itemsep.1em
    \item \textbf{Sim} -- \textsl{IncrementYears}, \textsl{AddNewStudents},
\textsl{AddNewGroups}.
    \end{enumerate}

\item In each academic month (August through April, inclusive):

    \begin{enumerate}
    \itemsep.1em
    \item Each \textbf{Student} -- \textsl{EncounterOthers},
\textsl{AdjustAttributes}, \textsl{DecayFriendships}. 
    \item Each \textbf{Group} -- \textsl{InfluenceMembers},
\textsl{RecruitStudents}, \textsl{LoseStudents}.
    \end{enumerate}

\item At the end of each academic year (post-April):

    \begin{enumerate}
    \itemsep.1em
    \item \textbf{Sim} -- \textsl{GraduateStudents}. % , \textsl{DropoutStudents}.
    \end{enumerate}
\end{enumerate}

\subsection{Initialization}

When the simulation begins, $n_{s_0}$ Student agents are created with unique
\textbf{ID}s, with \textbf{Year} $\thicksim \left \lfloor U(1,4) \right
\rfloor$ and $\mathbb{P}(\textbf{Race}=white) = p_w$. Each of $n_p$
\textbf{Preferences} and $n_h$ \textbf{Hobbies} are generated i.i.d.
$\thicksim U(0,1)$, and then the hobbies are normalized so that they sum to 1.

If policy \textbf{InitialMixedRaceDyads} is enabled, each Student is also
initially provided with $n_{d_0}$ friendships of randomly selected other
students of the opposite race.

Also at simulation start (regardless of policy), $n_{g_0}$ Groups are created
exactly as described in the \textsl{AddNewGroups} submodel, below.

If policy \textbf{OrientationGroups} is enabled, $n_{f_0}$ Groups are
created with unique \textbf{ID}s and with \textbf{isFixed} equal to TRUE. Each
of these groups $\mathcal{G}_k$, like those above, is assigned an initial
number of students $n_{sg_k} \thicksim \left\lfloor
U(\text{\textit{min}}_{sg},\text{\textit{max}}_{sg}) \right\rfloor$. These
students are deliberately chosen so that the proportion of minorities is
$\left\lfloor n_{sg_k} \cdot f_{\textit{minority}} \right \rfloor$. The
default value of $f_{\text{\textit{minority}}}=0.5$, considerably higher than
$(1-p_w)=0.2$, since the purpose of the policy is to expose white students to
greater numbers of minorities than they otherwise would encounter.

(All these numerical parameters are configurable. The simulation's default
values for them are $n_{s_0}=4000$, $p_w=0.8$, $n_p=20$, $n_h=20$,
$n_{d_0}=10$, $n_{g_0}=200$, $n_{f_0}=5$, and
$f_{\text{\textit{minority}}}=0.5$.)

\subsection{Submodels}

\begin{description}
\itemsep.8em
\item[\textsl{IncrementYears}] Add 1 to the \textbf{Year} attribute of every Student
still in the simulation.

\item[\textsl{AddNewStudents}] Add $n_s$ (default 1000) new Student agents to
the simulation with unique \textbf{ID}s, with \textbf{Year} = 1, and
$\mathbb{P}(\textbf{Race}=white) = p_w$. Each of $n_p$ \textbf{Preferences}
and $n_h$ \textbf{Hobbies} are generated i.i.d. $\thicksim U(0,1)$, and then
the hobbies are normalized so that they sum to 1.

\item[\textsl{AddNewGroups}] $n_g$ (default 10) Groups are created with unique
\textbf{ID}s and with \textbf{isFixed} equal to FALSE. Each Group
$\mathcal{G}_k$ is randomly assigned $n_{sg_k}$ initial students, where
$\forall k, 1\leq k\leq n_g, n_{sg_k} \thicksim \left\lfloor
U(\text{\textit{min}}_{sg},\text{\textit{max}}_{sg}) \right\rfloor$. Each
Group's \textbf{RecruitmentFactor} is generated $\thicksim U(0,1)$.

\item[\textsl{ComputeSimilarity}] Given two Student agents, compute their perceived
similarity as:

\[
w_r [\![r_1=r_2]\!] +
w_p \sum_i |p_{1i}-p_{2i}| + 
w_h \sum_i |h_{1i}-h_{2i}|
\]

where $w_r$, $w_p$, $w_h$ are weights given to race (default 20), preferences
(default 1.5), and hobbies (default 2.5), $[\![r_1=r_2]\!]$ is 1 if the students have the same
race and 0 otherwise, and $p_{ki}$ and $h_{ki}$ are the values of the $i$th
preference and $i$th hobby, respectively, of student $k$.

\item[\textsl{ComputeAffinity}] Given a Student and a Group, compute the mean
similarity between the Student and all individual members of the Group (using
submodel \textsl{ComputeSimilarity}).

\item[\textsl{EncounterOthers}] Choose $n_{eg}$ (default 10) other students at random
from the Student's current Groups, and $n_{ep}$ (default 5) from the population at
large. For each one, if the two Students are already friends, ``refresh" their
friendship. (See \textsl{DecayFriendships}, below.) If not, determine the
Students' similarity via the \textsl{ComputeSimilarity} submodel, and compute
$\mathbb{P}(\text{become friends})=f_c \cdot \textit{similarity} + f_i$, where
the default ``friendship coefficient" $f_c=0.22$ and ``friendship intercept"
$f_i=0.05$. (The rationale for this transformation is to establish a separate
baseline probability range for friendship rather than simply equating the raw
0-1 similarity measure with a probability. Using the similarity directly as a
probability would make friendship likelihood unreasonably high for similar
Students and unreasonably low for dissimilar.) With this probability, make the
two Students friends.

\item[\textsl{AdjustAttributes}] For each of a Student's preferences and
hobbies, compute the average value of that preference/hobby for all friends of
that Student. Then, with probability $p_a$ (default 0.1), adjust the Student's
attribute towards the mean by $d_a$ (default 0.2) multiplied by the difference between its current
value and the mean. (With probability $1-p_a$, make no such adjustment to that
attribute.)

\item[\textsl{InfluenceMembers}] Adjust the attribute values for each Student
in the Group, using the exact same algorithm as the \textsl{AdjustAttributes}
submodel except that instead of using the Students' friends' mean, use the
Group members' mean.

\item[\textsl{DecayFriendships}] Remove the friendship with each of the
Student's friends who has not been ``refreshed" (encountered while already
friends) in the most recent $t_d$ (default 2) academic months. This ``decay threshold"
models the maximum time two friends can be apart from one another and still
remain friends.

\item[\textsl{RecruitStudents}] Choose $n_r$ (default 10) Students from the general
population, and compute their \textit{affinity} to this Group via
\textsl{ComputeAffinity}. For each of the Students who are not already a
member of the Group, compute $\mathbb{P}(\text{join group})= \textit{affinity}
+ \textbf{RecruitmentFactor} - t_r$, where the ``recruitment threshold" is
$t_r$ (default 0.6). 
With this probability have the Student join the Group.

\item[\textsl{GraduateStudents}] Remove every Student with \textbf{Year}=4 from the
simulation.

\end{description}

The simulation proceeds for a fixed number of years $n_y$ (default 20), after which it terminates.
Throughout the simulation, text files are produced to capture relevant data
each academic month and year.

% I think we have no room; save for JASSS.
% 
% \subsection{Limitations and simplifications}
% 
% Finally, we report some known ways in which our model is known to depart from
% reality:
% 
% People are known to have varying strengths (or ``tiers") to their
% relationships \cite{hirshman_leaving_2011}. In fact, people have many weak
% ties and few strong ties. ((Granovetter 1973; Hill and Dunbar 2003; Zhou et
% al. 2005). For simplicity, we model all friendships as equal.
% 
% Interview and survey data indicates that minorities and whites differ on their
% segregation preferences. Whites, more than minorities, are more demanding of
% like-color neighbors, while a majority of African-Americans in major urban
% areas appear to prefer integrated neighborhoods
% \cite{farley_residential_1997}. Even this kind of ``unidirectional" prejudice
% has been demonstrated to produce wide-scale segregation
% \cite{chen_emergence_2005}. We do not model this asymmetry.
% 
% assume independence of attributes, though Block and Grund disproved it.
