
\section{THE MODEL}
\label{sec:model}

\subsection{Students}

% Axelrod 1997 -- culture is defined as ``a set of individual attributes that
% are subject to social influence."

\subsubsection{Attributes: preferences and hobbies}

\subsubsection{Attribute ``drift"}

\subsection{Groups}

\subsection{Policies}

\subsubsection{Forced integrated dyads}

\subsubsection{Forced integrated groups (``orientation groups")}


\subsection{Limitations and simplifications}

Known ways in which we depart from reality:

People are known to have varying strengths (or ``tiers") to their
relationships \cite{hirshman_leaving_2011}. In fact, people have many weak
ties and few strong ties. ((Granovetter 1973; Hill and Dunbar 2003; Zhou et
al. 2005). For simplicity, we model all friendships as equal.

Interview and survey data indicates that minorities and whites differ on their
segregation preferences. Whites, more than minorities, are more demanding of
like-color neighbors, while a majority of African-Americans in major urban
areas appear to prefer integrated neighborhoods
\cite{farley_residential_1997}. Even this kind of ``unidirectional" prejudice
has been demonstrated to produce wide-scale segregation
\cite{chen_emergence_2005}. We do not model this asymmetry.

assume independence of attributes, though Block and Grund disproved it.
